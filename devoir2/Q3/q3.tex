\section{Question 3}

\subsection{Jacobi}
% TODO comparer itérations
La figure~\ref{fig:ar} nous donne un aperçu de l'impact de $r$
sur l'erreur pour différentes valeurs de $a$.

On voit que pour $a \geq 0.3$, la régularisation donne de meilleurs résultats
pour autant qu'on choisissse un bon $r$.
Pour $a \leq 0.25$ par contre, la régularisation donne de moins bons résultats.

On a d'ailleurs pas toujours la convergence pour $a \geq 0.25$ mais à partir
d'un certain $r$, ça converge.
Le meilleur $r$ pour $a \geq 0.3$ semble être le premier $r$ à partir duquel
ça converge.
Pour $a \leq 0.25$, ça n'a pas l'air aussi simple mais la fonction semble
unimodale donc on peut aisément trouver le $r$ qui minimise l'erreur.

\begin{figure}
  \centering
  \includegraphics[width=\textwidth]{Q3/arJacobi.png}
  \caption{Erreur pour différents $a$ en fonction de $r$ de Jacobi.
  Lorsque la méthode ne converge pas, le point n'est pas représenté.
  L'erreur est calculée comme la norme de Frobenius entre l'image de départ
  et l'image défloutée. Les croix représentent quant à elles l'erreur obtenue pour différent a à la question 2 (identique pour Jacobi et Gauss-Seidel).}
  \label{fig:ar}
\end{figure}

\subsection{Gauss-Seidel}

\begin{figure}
  \centering
  \includegraphics[width=\textwidth]{Q3/arGauss-Seidel.png}
  \caption{Erreur pour différents $a$ en fonction de $r$ de Gauss-Seidel.
  Lorsque la méthode ne converge pas, le point n'est pas représenté.
  L'erreur est calculée comme la norme de Frobenius entre l'image de départ
  et l'image défloutée. Les croix représentent quant à elles l'erreur obtenue pour différent a à la question 2 (identique pour Jacobi et Gauss-Seidel).}
  \label{fig:ar}
\end{figure}

\begin{table}
  \centering
  \begin{tabular}{|l|l|l|l|l|}
    \hline
    \multirow{2}{*}{$a$} & \multicolumn{2}{l|}{Jacobi} & \multicolumn{2}{l|}{Gauss-Seidel}\\
    \cline{2-5}
        & $i_1$ & $i_2$ & $i_1$ & $i_2$\\
    \hline
    0.2 &     &     &     & \\
    \hline
    0.4 & 71    &     &     & \\
    \hline
    0.45&   &    &     & \\
    \hline
    0.5 &    &   &    & \\
    \hline
  \end{tabular}
  \caption{Nombre d'itération nécessaire pour déflouter pour différentes valeurs de $a$ (le défloutage abandonne à 500 itérations).
  $i_1$ est le nombre d'itérations pour le premier système et $i_2$ pour le deuxième.}
  \label{tab:iter}
\end{table}

\subsubsection{Comparaison entre Gauss-Seidel et Jacobi}
En comparant la figure~\ref{fig:arj} et la figure~\ref{fig:args}, on voit
que lorsque Jacobi converge, ils ont la même erreur.
Seulement, Jacobi ne converge pas toujours pour le meilleur $r$ pour Gauss-Seidel.
C'est d'ailleurs assez naturel qu'ils aient la même erreur car ils résolvent le même
système linéaire.
C'est donc pour ça qu'on a pas comparé l'erreur à la question 2.

\subsection{Analyse du rayon spectral}

Comme illustré aux figures $\ref{fig:Jrayon}$ et $\ref{fig:GSrayon}$, le rayon spectral tend à diminuer pour un r de plus en plus grand. On constate également que la méthode de la question 3 n'a un rayon spectral plus petit qu'à la question 2 qu'à partir d'un certain r. La régularisation n'est donc intéressante au point de vue de la vitesse qu'à partir d'un certain r. Par exemple pour $a = 0.3$, il faut choisir un minimum de $r = 0.5$ pour que la méthode de Gauss-Seidel converge plus rapidement. 


\begin{figure} \label{fig:Jrayon}
  \centering
  \includegraphics[width=\textwidth]{Q3/rhoJacobi.png}
  \caption{Graphe du rayon spectral de A $\rho(A)$ en fonction de r et ce pour différentes valeurs de a pour la méthode de Jacobi. Les croix indiquent quant à elles les valeurs du rayon spectral obtenues à la question 2.}
  \label{fig:ar}
\end{figure}

 \begin{figure} \label{fig:GSrayon}
  \centering
  \includegraphics[width=\textwidth]{Q3/rhoGauss-Seidel.png}
  \caption{Graphe du rayon spectral de A $\rho(A)$ en fonction de r et ce pour différentes valeurs de a pour la méthode de Gauss-Seidel. Les croix indiquent quant à elles les valeurs du rayon spectral obtenues à la question 2.}
  \label{fig:ar}
\end{figure}

\subsection{Recherche unimodale}
Bien qu'on pourrait faire une recherche unimodale directement, on prenant
$+\infty$ comme erreur par convention quand ça ne converge pas,
commençons par faire une recherche binaire pour trouver à partir de quel
$r$ ça converge.

Faisons ensuite une recherche unimodale du minimum
avec $r$ entre cette valeur et 0.5.
En effet, comme vu à la figure~\ref{fig:arj} et à la figure~\ref{fig:args},
le minimum se situe toujours entre 0 et 0.5.

On obtient la figure~\ref{fig:best} où on observe les même minimums que sur la figure~\ref{fig:arj}
et la figure~\ref{fig:args}.
On voit que la valeur de $r$ est à peu prêt la même pour Gauss-Seidel et Jacobi,
c'est normal car leur erreur en fonction de $r$ est pareil comme ils résolvent le même système.
Ce n'est par contre plus le cas pour $a \geq 0.25$ car Jacobi ne converge alors plus pour tout $r$.
Dans ce cas, le minimum est atteint pour le premier $r$ tel qu'il converge.

\begin{figure}
  \centering
  \includegraphics[width=\textwidth]{Q3/best.png}
  \caption{Valeur de $r$ donnant la plus petite erreur pour différentes valeurs de $a$.}
  \label{fig:best}
\end{figure}
