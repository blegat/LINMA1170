\section{Question 1}

\subsection{Facteur k}
	Tout d'abord, il  est capital que peu importe la valeur de $\textit{a}$ dans la matrice $\textit{T}$, les coefficients de la matrice $\textit{A}$ gardent leur valeur comprise entre 0 et 255. La valeur de $\frac{1}{k}$ doit donc permettre de ramener les valeurs obtenues après floutage dans des normes acceptables.
	
\textbf{Il y a clairement une autre utilité, preuve en est si on fait joujou avec le programme matlab on se rend compte qu'à a constant modifier k a un impact clair sur l'image après recomposition.}	
	
\subsection{Valeurs propres}
	Nous savons que les valeurs propres de la matrice S 
	$$
S = 	
 \left[
 \begin{array}{cccc}
    0 & 1 		& 			& \\
    1 & \ddots 	& \ddots 	& \\
      & \ddots 	& \ddots 	& 1 \\
      & 		& 1			& 0
  \end{array}
  \right]
$$

sont données par $\lambda_i = 2 \cos(\frac{i\pi}{n+1})$, $i= 1, \ldots , n$ où n est la dimension de la matrice.

Vu que T est de la forme 

$$
 T = \frac{1}{k}
 \left[
 \begin{array}{cccc}
    1 & a 		& 			& \\
    a & \ddots 	& \ddots 	& \\
      & \ddots 	& \ddots 	& a \\
      & 		& a			& 1
  \end{array}
  \right] ,
$$

on a donc que $T = \frac{1}{k} I + \frac{a}{k} S$.

%XXX
\textbf{je sais je dois encore justifier proprement}

Les valeurs propres de T sont donc du type $\lambda_i(T) = \frac{1}{k} + 2 \frac{a}{k} \cos(\frac{i\pi}{n+1})$.

On obtient donc comme valeurs propres maximales $\lambda_{max}$ et minimum $\lambda_{min}$ , pour a et k positifs (on travaille avec des pixels ayant leur valeur comprise entre 0 et 255): 

	$$\lambda_{max} (T) =  \frac{1+2a}{k} $$
	$$\lambda_{min} (T) =  \frac{1-2a}{k} $$
	
Le nombre de conditionnement $\kappa$ de la matrice $T$ (symétrique) est défini par 
\begin{equation}
	\kappa (T) = ||T||_2||T^{-1}||_2 = \frac{\lambda_{max} (T)}{\lambda_{min} (T)} = \frac{1+2a}{1-2a}
\end{equation}

La norme subordonnée $||T||_2$ est définie telle que : 
$$||T||_2 = \sqrt{\lambda_{max}(T^TT)}$$
où $\lambda_{max}(T^TT)$ désigne la plus grande valeur propre de $T^TT$.
On sait donc que $\kappa(T) $ est positif.	

\textbf{Parait même que c'est $\geq$ 1}

Le carré du nombre de conditionnement est :
\begin{equation}
	\kappa^2 (T) = \frac{(1+2a)^2}{(1-2a)^2}
\end{equation}


(On dit d'une matrice qu'elle est bien conditionnée si son nombre de conditionnement est proche de 1 et mal conditionnée s'il est grand. En effet, dans le cadre d'un problème de type $Ax = b$, la valeur de ce nombre fournit une borne de l'erreur relative de $x$ quand on introduit une perturbation $\Delta A $ ou $\Delta b$. Dans notre cas on cherchera donc à prendre un $a$ le plus proche possible de 0. )

On remarque également que pour $a \geq 0.5$ le nombre de conditionnement n'est pas défini. En $0.5$ de manière évidente et pour $a > 0.5$ parce que $\kappa$ est positif. 
	

