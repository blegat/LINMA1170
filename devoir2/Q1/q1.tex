\section{Question 1}

\subsection{Facteur k}
	Tout d'abord, il  est capital que peu importe la valeur de $\textit{a}$ dans la matrice $\textit{T}$, les coefficients de la matrice $\textit{A}$ gardent leur valeur comprise entre 0 et 255. La valeur de $\frac{1}{k}$ doit donc permettre de ramener les valeurs obtenues après floutage dans des normes acceptables.
	
	
\subsection{Valeurs propres}
	Nous savons que les valeurs propres de la matrice S 
	$$
S = 	
 \left[
 \begin{array}{cccc}
    0 & 1 		& 			& \\
    1 & \ddots 	& \ddots 	& \\
      & \ddots 	& \ddots 	& 1 \\
      & 		& 1			& 0
  \end{array}
  \right]
$$

sont données par $\lambda_i = 2 \cos(\frac{i\pi}{n+1})$, $i= 1, \ldots , n$ où n est la dimension de la matrice.

Vu que T est de la forme 

$$
 T = \frac{1}{k}
 \left[
 \begin{array}{cccc}
    1 & a 		& 			& \\
    a & \ddots 	& \ddots 	& \\
      & \ddots 	& \ddots 	& a \\
      & 		& a			& 1
  \end{array}
  \right] ,
$$

on a donc que $T = \frac{1}{k} I + \frac{a}{k} S$.

%XXX
\textbf{je sais je dois encore justifier proprement}

Les valeurs propres de T sont donc du type $\lambda_i(T) = \frac{1}{k} + 2 \frac{a}{k} \cos(\frac{i\pi}{n+1})$.

On obtient donc comme valeurs propres maximales $\lambda_{max}$ et minimum $\lambda_{min}$ , pour a et k positifs : 

	$$\lambda_{max} (T) =  \frac{1+2a}{k} $$
	$$\lambda_{min} (T) =  \frac{1-2a}{k} $$
	
Le nombre de conditionnement $\kappa$ de la matrice $T$ est défini par 
\begin{equation}
	\kappa (T) = \frac{\lambda_{max} (T)}{\lambda_{min} (T)} = \frac{1+2a}{1-2a}
\end{equation}
	
Le carré du nombre de conditionnement est :
\begin{equation}
	\kappa^2 (T) = \frac{(1+2a)^2}{(1-2a)^2}
\end{equation}
	

